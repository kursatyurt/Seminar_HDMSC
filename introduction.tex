\section{Introduction}

Ever-increasing internet speed increased the data produced worldwide, which led to a boom in Machine Learning and Data Analytics fields.
Those areas generally have many dimensional large data sets to handle. As with all high-dimensional problems, they suffer from the curse of dimensionality, i.e., \ they have an
exponential dependency on dimension. This is a barrier to the numerical treatment of high-dimensional problems.
This exponential dependency makes it harder to use classical mesh-based approaches to solve this problem. One could also use mesh-free methods like Monte-Carlo quadratures.

In order to overcome such a problem, the sparse grid method gains more and more popularity.
The sparse grid method is a general numerical discretization technique first introduced by the Russian mathematician Smolyak in 1963~\cite{smolyak1963quadrature}.

In this paper we will outlook into how to construct a classical sparse grid in section~\ref{sec:sparsegrid}, and look at the adaptivity of the sparse grid method in section~\ref{sec:adaptivity}.
We will also look at the use of the spatially adaptive sparse grid method in the context of different test functions in section~\ref{sec:example}.
A final conclusion and outlook will be drawn in section~\ref{sec:conclusion}.