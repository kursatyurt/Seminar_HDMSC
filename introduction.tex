\section{Introduction}

Ever increasing internet speed and data production and collection come with some new challenges to computational science.
Beside the classical high dimensional problem like finance, some new player are also jumped to the train like machine learning with an immense amount of dimension.
High dimensional problems are suffer from curse of dimensionality, i.e.\ they have an exponential dependency on dimension. This is a barrier in numerical treatment of the high-dimensonal problems.
This exponential dependency makes harder to use classical mesh based approaches to solve this kind of problems. One could also use mesh-free methods like Monte-Carlo quadratures.

In order to overcome such a problem, the sparse grid method gains more and more popularity.
The sparse grid method is a general numerical discretization technique which is first introduced by  the Russian mathematician Smolyak in 1963~\cite{smolyak1963quadrature}.

Sparse grids offers a new way to reduce the required number of grid points by the order of magnitude \(O(2^{nd})\) to just only \(O(2^n n^{d-1})\) while preseving a similar error as using the full grid.
In order to achieve these bounds, the mixed second derivatives have to be bounded \todok{Find reference for this.}. For the problems which are do not satisfy smoothness criteria or required further reduce in meshs size, one can use advantage of adaptivity.
The hierarchical basis is a direct indicator of areas where further refinement required.

\todok{Sparse grids are based on a hierarchical (and thus inherently incremental and adaptive)
    formulation of the one-dimensional basis which is then extended to the d-dimensional
    setting via a tensor product approach.}

\missingfigure{Add comparision of nodal basis vs hierarchical basis}

\subsection{Comparision to Full Grids}

\begin{table}
    \centering
    \begin{tabular}{l c c}
        \multicolumn{1}{c}{} & Storage Requirement  & \multicolumn{1}{c}{L2 Norm of Interpolation Error} \\
        \toprule
        Full Grid            & \(O(2^{nd})\)        & \(O(2^{-2n})\)                                     \\
        \midrule
        Sparse Grid          & \(O(2^{n} n^{d-1})\) & \(O(2^{-2n} n^{d-1})\)                             \\
        \bottomrule
    \end{tabular}
\end{table}

\missingfigure{Show full grid vs Hierarchy in 1D}

